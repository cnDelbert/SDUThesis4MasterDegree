% !Mode:: "TeX:UTF-8"
\begin{thebibliography}{99}
\addcontentsline{toe}{chapter}{References}
\bibitem{LeCun1989}LeCun, Yann, Boser, Bernhard, Denker, John S, Henderson, Donnie, Howard, Richard E, Hubbard, Wayne, and Jackel, Lawrence D.  Backpropagation applied to handwritten zip code recognition. Neural computation , 1(4):541–551, 1989.
\bibitem{MIRResAndPro2008SJH}孙吉红,刘伟成,焦玉英,多媒体信息检索研究与展望,计算机应用研究,2008, 25(3):646-649.
\bibitem{PR1984ImageSystems}Tamura, Hideyuki, and Naokazu Yokoya. Image database systems: A survey, Pattern recognition, 1984, 17(1): 29-43.
\bibitem{KwRtWWW1998}La Cascia M, Sethi S, Sclaroff S. Combining textual and visual cues for content-based image retrieval on the world wide web[C]//Content-Based Access of Image and Video Libraries, 1998. Proceedings. IEEE Workshop on. IEEE, 1998: 24-28.
\bibitem{CLEF2013Overview}Mauricio V., Roberto P., and Bart Thomee. Overview of the ImageCLEF 2013 Scalable Concept Image Annotation Subtask. CLEF 2013 Evaluation Labs and Workshop, Online Working Notes, Valencia, Spain, September 23-26, 2013.
\bibitem{kato1992database}Kato, Toshikazu. Database architecture for content-based image retrieval. SPIE/IS\&T 1992 Symposium on Electronic Imaging: Science and Technology. International Society for Optics and Photonics, 1992.
% {Mauricio Villegas and Roberto Paredes and Bart Thomee},
%   title = {{Overview of the ImageCLEF 2013 Scalable Concept Image Annotation Subtask}},
%   booktitle = {CLEF 2013 Evaluation Labs and Workshop, Online Working Notes},
%   year = {2013},
%   month = {September 23-26},
%   address = {Valencia, Spain}
\bibitem{Van2008tsne}Van der Maaten, Laurens, and Geoffrey Hinton. Visualizing data using t-SNE. Journal of Machine Learning Research 9.2579-2605 (2008): 85.

\bibitem{smeulders2000content}Smeulders, A. W., Worring, M., Santini, S., Gupta, A., Jain, R. (2000). Content-based image retrieval at the end of the early years. Pattern Analysis and Machine Intelligence, IEEE Transactions on, 22(12), 1349-1380.
\bibitem{MultiXie2012}谢毓湘, 栾悉道, and 吴玲达. 多媒体数据语义鸿沟问题分析. 武汉理工大学学报: 信息与管理工程版 33.6 (2012): 859-863.
\bibitem{hanbury2008survey}Hanbury, Allan. A survey of methods for image annotation. Journal of Visual Languages \& Computing 19.5 (2008): 617-627.
\bibitem{carson2002blobworld}Carson, Chad, et al. Blobworld: Image segmentation using expectation-maximization and its application to image querying. Pattern Analysis and Machine Intelligence, IEEE Transactions on 24.8 (2002): 1026-1038.
\bibitem{lowe1999sift}Lowe, David G. Object recognition from local scale-invariant features. Computer vision, 1999. The proceedings of the seventh IEEE international conference on. Vol. 2. Ieee, 1999.
\bibitem{kherfi2004image}Kherfi, Mohammed Lamine, Djemel Ziou, and Alan Bernardi. Image retrieval from the world wide web: Issues, techniques, and systems. ACM Computing Surveys (CSUR) 36.1 (2004): 35-67.
\bibitem{Book2012wujun}吴军. 数学之美. Vol. 5. 人民邮电出版社, 2012.
\bibitem{wang2009simultaneous}Wang, Chong, David Blei, and Fei-Fei Li. Simultaneous image classification and annotation. Computer Vision and Pattern Recognition, 2009. CVPR 2009. IEEE Conference on. IEEE, 2009. % 基于统计的
\bibitem{guillaumin2009tagprop}Guillaumin, Matthieu, et al. Tagprop: Discriminative metric learning in nearest neighbor models for image auto-annotation. Computer Vision, 2009 IEEE 12th International Conference on. IEEE, 2009. % 概率统计的
\bibitem{villegas2009image}Villegas, Mauricio, and Roberto Paredes. Image-Text Dataset Generation for Image Annotation and Retrieval.Prometeo (2009): 014.
\bibitem{villegas2012overview}Villegas, Mauricio, and Roberto Paredes. Overview of the ImageCLEF 2012 Scalable Web Image Annotation Task. CLEF (Online Working Notes/Labs/Workshop). 2012.
\bibitem{ICML2011Ngiam}J. Ngiam, A. Khosla, M. Kim, J. Nam, H. Lee, and A. Y. Ng, Multimodal Deep Learning, International Conference on Machine Learning, 2011, pp. 689-696.
\bibitem{de2001pattern}De Sa, JP Marques. Pattern recognition: concepts, methods, and applications. Springer Science \& Business Media, 2001.
\bibitem{Yang2013bp}杨守建, 陈恳. BP 神经网络性能与隐藏层结构的相关性探究. 宁波大学学报: 理工版 (2013): 48-52.
\bibitem{Wang2008Bpgai}王燕. 一种改进的 BP 神经网络手写体数字识别方法. 计算机工程与科学 30.4 (2008): 50-52.
\bibitem{Lv2001bp}吕砚山, 赵正琦. BP 神经网络的优化及应用研究[J]. 北京化工大学学报, 2001, 28(1): 67-69.
\bibitem{donahue2013decaf}Donahue, Jeff, Yangqing Jia, Oriol Vinyals, Judy Hoffman, Ning Zhang, Eric Tzeng, and Trevor Darrell. Decaf: A deep convolutional activation feature for generic visual recognition. arXiv preprint arXiv:1310.1531 (2013).
\bibitem{HuadongligongZhaoyaxin}赵雅昕,图像内容表示及多标签标注算法研究,华东理工大学硕士学位论文,2013.
\bibitem{WuhanligongLijing}李静,基于多特征的图像标注研究,武汉理工大学硕士学位论文,2013.
\bibitem{Youmei2015PR}Wei Zhang, Youmei Zhang, Lin Ma, Jingwei Guan, Shijie Gong. Multimodal learning for facial expression recognition, Pattern Recognition, 2015.
\bibitem{HuazhongkedaZhouquan}周全,基于上下文的图像标注研究,华中科技大学博士学位论文,2013.
\bibitem{YangqingjiaCaffe}Jia, Y. Caffe: An open source convolutional architecture for fase feature embedding. \url{http://caffe.berkeleyvision.org}, 2013.
\bibitem{krizhevsky2012imagenet}Krizhevsky, Alex, Ilya Sutskever, and Geoffrey E. Hinton. Imagenet classification with deep convolutional neural networks. In Advances in neural information processing systems, pp. 1097-1105. 2012.
\bibitem{zeiler2014visualizing}Zeiler, Matthew D., and Rob Fergus. Visualizing and understanding convolutional networks. In Computer Vision–ECCV 2014, pp. 818-833. Springer International Publishing, 2014.
\bibitem{szegedy2014going}Szegedy, Christian, Wei Liu, Yangqing Jia, Pierre Sermanet, Scott Reed, Dragomir Anguelov, Dumitru Erhan, Vincent Vanhoucke, and Andrew Rabinovich. Going deeper with convolutions. arXiv preprint arXiv:1409.4842 (2014).
\bibitem{simonyan2014verydeep}Simonyan, Karen, and Andrew Zisserman. Very deep convolutional networks for large-scale image recognition. arXiv preprint arXiv:1409.1556 (2014).
\bibitem{ILSVRC15}Olga Russakovsky and Jia Deng and Hao Su and Jonathan Krause and Sanjeev Satheesh and Sean Ma and Zhiheng Huang and Andrej Karpathy and Aditya Khosla and Michael Bernstein and Alexander C. Berg and Li Fei-Fei. Imagenet large scale visual recognition challenge. arXiv preprint arXiv:1409.0575 (2014).
\bibitem{DianzikedaLuzuyou}卢祖友,图像语义标注方法研究及其系统实现,电子科技大学硕士学位论文,2009.
\end{thebibliography}


% [1] http://www.ifr.org/service-robots/ 
% [1]蔡自兴. 机器人学[M]. 北京:清华大学出版社.2000.
% [2]许国. 自动化立体仓库控制技术应用研究. 中国科学技术大学工学硕士学位论文. 2007.5.
% [3]张志喜. J公司核心竞争力构建研究. 天津师范大学硕士学位论文.2012.5.
% [4]Rachkov, M.Optimal control of the transport robot [J]. Industrial Electronics, Control and Instrumentation,1994(2) : 1039-1042.
% [5]丁良宏,王润孝,冯华山,李军.浅析BigDog四足机器人[J]. 中国机械工程,2011,23(5):505-514.
% [6]http://spectrum.ieee.org/robotics/robotics-software/three-engineers-hundreds-of-robots-one-warehouse/0.
% [7]Tatsuo Sakai,Anselm Lim Yi Xiong. Image-Processing technologies for service robot “HOSPI-Rimo” [J].Panasonic Technical Journal, 2013.58(4): 307-309